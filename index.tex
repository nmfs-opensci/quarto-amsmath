% Options for packages loaded elsewhere
\PassOptionsToPackage{unicode}{hyperref}
\PassOptionsToPackage{hyphens}{url}
\PassOptionsToPackage{dvipsnames,svgnames,x11names}{xcolor}
%
\documentclass[
  letterpaper,
  DIV=11,
  numbers=noendperiod]{scrreprt}

\usepackage{amsmath,amssymb}
\usepackage{iftex}
\ifPDFTeX
  \usepackage[T1]{fontenc}
  \usepackage[utf8]{inputenc}
  \usepackage{textcomp} % provide euro and other symbols
\else % if luatex or xetex
  \usepackage{unicode-math}
  \defaultfontfeatures{Scale=MatchLowercase}
  \defaultfontfeatures[\rmfamily]{Ligatures=TeX,Scale=1}
\fi
\usepackage{lmodern}
\ifPDFTeX\else  
    % xetex/luatex font selection
\fi
% Use upquote if available, for straight quotes in verbatim environments
\IfFileExists{upquote.sty}{\usepackage{upquote}}{}
\IfFileExists{microtype.sty}{% use microtype if available
  \usepackage[]{microtype}
  \UseMicrotypeSet[protrusion]{basicmath} % disable protrusion for tt fonts
}{}
\makeatletter
\@ifundefined{KOMAClassName}{% if non-KOMA class
  \IfFileExists{parskip.sty}{%
    \usepackage{parskip}
  }{% else
    \setlength{\parindent}{0pt}
    \setlength{\parskip}{6pt plus 2pt minus 1pt}}
}{% if KOMA class
  \KOMAoptions{parskip=half}}
\makeatother
\usepackage{xcolor}
\setlength{\emergencystretch}{3em} % prevent overfull lines
\setcounter{secnumdepth}{5}
% Make \paragraph and \subparagraph free-standing
\makeatletter
\ifx\paragraph\undefined\else
  \let\oldparagraph\paragraph
  \renewcommand{\paragraph}{
    \@ifstar
      \xxxParagraphStar
      \xxxParagraphNoStar
  }
  \newcommand{\xxxParagraphStar}[1]{\oldparagraph*{#1}\mbox{}}
  \newcommand{\xxxParagraphNoStar}[1]{\oldparagraph{#1}\mbox{}}
\fi
\ifx\subparagraph\undefined\else
  \let\oldsubparagraph\subparagraph
  \renewcommand{\subparagraph}{
    \@ifstar
      \xxxSubParagraphStar
      \xxxSubParagraphNoStar
  }
  \newcommand{\xxxSubParagraphStar}[1]{\oldsubparagraph*{#1}\mbox{}}
  \newcommand{\xxxSubParagraphNoStar}[1]{\oldsubparagraph{#1}\mbox{}}
\fi
\makeatother


\providecommand{\tightlist}{%
  \setlength{\itemsep}{0pt}\setlength{\parskip}{0pt}}\usepackage{longtable,booktabs,array}
\usepackage{calc} % for calculating minipage widths
% Correct order of tables after \paragraph or \subparagraph
\usepackage{etoolbox}
\makeatletter
\patchcmd\longtable{\par}{\if@noskipsec\mbox{}\fi\par}{}{}
\makeatother
% Allow footnotes in longtable head/foot
\IfFileExists{footnotehyper.sty}{\usepackage{footnotehyper}}{\usepackage{footnote}}
\makesavenoteenv{longtable}
\usepackage{graphicx}
\makeatletter
\def\maxwidth{\ifdim\Gin@nat@width>\linewidth\linewidth\else\Gin@nat@width\fi}
\def\maxheight{\ifdim\Gin@nat@height>\textheight\textheight\else\Gin@nat@height\fi}
\makeatother
% Scale images if necessary, so that they will not overflow the page
% margins by default, and it is still possible to overwrite the defaults
% using explicit options in \includegraphics[width, height, ...]{}
\setkeys{Gin}{width=\maxwidth,height=\maxheight,keepaspectratio}
% Set default figure placement to htbp
\makeatletter
\def\fps@figure{htbp}
\makeatother

\usepackage{mathtools}
\KOMAoption{captions}{tableheading}
\makeatletter
\@ifpackageloaded{bookmark}{}{\usepackage{bookmark}}
\makeatother
\makeatletter
\@ifpackageloaded{caption}{}{\usepackage{caption}}
\AtBeginDocument{%
\ifdefined\contentsname
  \renewcommand*\contentsname{Table of contents}
\else
  \newcommand\contentsname{Table of contents}
\fi
\ifdefined\listfigurename
  \renewcommand*\listfigurename{List of Figures}
\else
  \newcommand\listfigurename{List of Figures}
\fi
\ifdefined\listtablename
  \renewcommand*\listtablename{List of Tables}
\else
  \newcommand\listtablename{List of Tables}
\fi
\ifdefined\figurename
  \renewcommand*\figurename{Figure}
\else
  \newcommand\figurename{Figure}
\fi
\ifdefined\tablename
  \renewcommand*\tablename{Table}
\else
  \newcommand\tablename{Table}
\fi
}
\@ifpackageloaded{float}{}{\usepackage{float}}
\floatstyle{ruled}
\@ifundefined{c@chapter}{\newfloat{codelisting}{h}{lop}}{\newfloat{codelisting}{h}{lop}[chapter]}
\floatname{codelisting}{Listing}
\newcommand*\listoflistings{\listof{codelisting}{List of Listings}}
\makeatother
\makeatletter
\makeatother
\makeatletter
\@ifpackageloaded{caption}{}{\usepackage{caption}}
\@ifpackageloaded{subcaption}{}{\usepackage{subcaption}}
\makeatother

\ifLuaTeX
  \usepackage{selnolig}  % disable illegal ligatures
\fi
\usepackage{bookmark}

\IfFileExists{xurl.sty}{\usepackage{xurl}}{} % add URL line breaks if available
\urlstyle{same} % disable monospaced font for URLs
\hypersetup{
  pdftitle={Quatro for amsmath LaTeX users},
  colorlinks=true,
  linkcolor={blue},
  filecolor={Maroon},
  citecolor={Blue},
  urlcolor={Blue},
  pdfcreator={LaTeX via pandoc}}


\title{Quatro for amsmath LaTeX users}
\author{}
\date{}

\begin{document}
\maketitle

\renewcommand*\contentsname{Table of contents}
{
\hypersetup{linkcolor=}
\setcounter{tocdepth}{2}
\tableofcontents
}

\bookmarksetup{startatroot}

\chapter*{Preface}\label{preface}
\addcontentsline{toc}{chapter}{Preface}

\markboth{Preface}{Preface}

\emph{tldr; For LaTeX users wanting complex math in both html and PDF:
1) Don't mix Quarto equation and amsmath syntax. Stick with amsmath. 2)
Add a mathjax.html file and add to your top yaml to turn-on equation
numbering and only use LaTeX equation environments (not \$\$). 2) For
referencing, only use \texttt{\textbackslash{}label\{\}} and
\texttt{\textbackslash{}eqref\{\}} or \texttt{\textbackslash{}ref\{\}}.
Do not use \texttt{(\#eq-)} and \texttt{@eq-} for cross-refs. Pandoc
(and MathJax) knows LaTeX so all your fancy equations should work if you
do 1 and 2.}

These are notes for those creating a Quarto book that they want to
display and html and PDF formats. Those only needing Word format, sorry
not sure that works as of July 2023; I have not tested since then.

\subsection*{To do}\label{to-do}
\addcontentsline{toc}{subsection}{To do}

\begin{itemize}
\tightlist
\item
  Adding equations that look like \texttt{3.1} so with the chapter
  numbering. We need this in Quarto books. MathJax
  \href{https://docs.mathjax.org/en/latest/input/tex/extensions/tagformat.html\#tex-tagformat}{tagformat}
  looks like the way to go.
\item
  Going forward, I think we might need a lua filter for LaTeX docs.
  Perhaps this already exists? Like, we should have to re-write a basic
  LaTeX doc that has \texttt{\textbackslash{}section\{abs\}}. We should
  be able to include our LaTeX chapters (\texttt{chap1.tex}) directly in
  the Quarto books and have standard LaTeX code with R or Python code
  mixed in.
\end{itemize}

\subsection*{Contributing!}\label{contributing}
\addcontentsline{toc}{subsection}{Contributing!}

Please put in a pull request and add more examples. Also Quarto keeps
updating so these notes will get out of date. I used Quarto 1.5.55 and
output from \texttt{pandoc\ -\/-version} is

\begin{verbatim}
pandoc 2.19
Compiled with pandoc-types 1.22.2, texmath 0.12.5.2, skylighting 0.13,
citeproc 0.8.0.1, ipynb 0.2, hslua 2.2.1
Scripting engine: Lua 5.4
\end{verbatim}

{(\setSection{1})}

\bookmarksetup{startatroot}

\chapter{Background}\label{background}

{(\nextSection)}

The default way to number and cross-ref equation numbers in Quarto is a
\texttt{\$\$} fence with \texttt{\{\#eq-\}}:

\begin{verbatim}
$$
S_n = \frac{X_1 + X_2 + \cdots + X_n}{n}
      = \frac{1}{n}\sum_{i}^{n} X_i
$$ {#eq-eq1}
\end{verbatim}

\begin{equation}\phantomsection\label{eq-eq1}{
S_n = \frac{X_1 + X_2 + \cdots + X_n}{n}
      = \frac{1}{n}\sum_{i}^{n} X_i
}\end{equation}

Then you cross-ref with \texttt{@eq-eq1} which gives you this:
Equation~\ref{eq-eq1}. It is works great with single equation in display
math\ldots and it causes all sorts of problems for LaTeX users.

The problem with this for LaTeX users is that if you are a LaTeX user
then you are using (almost certainly) amsmath and the more complex
equation environments and equation numbers. That Quarto syntax above is
not LaTeX and going to break things for you on many levels (Pandoc,
mathjax) for things that expect proper amsmath syntax. Also you do not
want to be re-writting your LaTeX equation code for 100s of equations
from your LaTeX docs!!

Rather than try to get Quarto syntax to work with your AMS equation
environments, I suggest never mixing Quarto syntax and amsmath syntax.
\textbf{Only} use proper amsmath syntax. So no \texttt{\$\$} fencing
around your
\texttt{\textbackslash{}begin\{\textless{}environment\textgreater{}\}}
stuff. That \texttt{\$\$} fencing is improper syntax and using it, while
seemingly a shortcut to get what you need, is going to make you suffer
later.

\bookmarksetup{startatroot}

\chapter{yaml for your Quarto book}\label{yaml-for-your-quarto-book}

{(\nextSection)}

MathJax is handling the display of your equations in html and you need
to set the configuration to show numbers. Oddly, that is not the
default. So you need \texttt{include-in-header:\ mathjax.html} in your
Quarto book yaml. See the mathjax tab for your \texttt{mathjax.html}
code.

This is what a minimal \texttt{\_quarto.yaml} for numbered equations
looks like

\begin{verbatim}
project:
  type: book

book:
  title: "Quatro for amsmath LaTeX users"
  chapters:
    - index.qmd
    - yaml.qmd
    - numbered_equations.qmd

format:
  html:
    include-in-header: mathjax.html
  pdf:
    include-in-header:
      - text: |
          \usepackage{mathtools}
# Don't use this. Breaks numbering (Aug 2024)
#    html-math-method:
#      method: mathjax
#      url: "https://mathjax.rstudio.com/latest/MathJax.js?config=TeX-AMS-MML_HTMLorMML"
\end{verbatim}

August 2024: I don't know why that url is not working. It was in Jan
2024. It is just setting the MathJax version. Probably the url is out of
date somehow.

\bookmarksetup{startatroot}

\chapter{Basic numbered equations}\label{basic-numbered-equations}

{(\nextSection)}

\emph{Reminder: Do not use
\href{https://quarto.org/docs/authoring/cross-references.html\#equations}{Quarto
equation numbering syntax}. Stay with proper amsmath LaTeX. Quarto
syntax is fine if you only have single equations in display math, but
you are on this page because that is not what you need. You need the
full amsmath capabilities.}

\textbf{Yes}, the equations don't have 3.1, 3.2 like they should in html
output even though they are fine in PDF. Still trying to figure out how
to get the Quarto chapter variable over to MathJax tagformat extension
which can add the chapter labels.

\textbf{Important} All the LaTeX is included plain in the qmd so none of
this type of fencing:

Wiki showing examples
\href{https://en.wikibooks.org/wiki/LaTeX/Advanced_Mathematics\#Other_environments}{advanced
math}

If you want one equation number, then use
\texttt{\textbackslash{}begin\{equation\}\textbackslash{}end\{equation\}}
fence. You can use
\texttt{\textbackslash{}begin\{gathered\}\textbackslash{}end\{gathered\}}
or
\texttt{\textbackslash{}begin\{aligned\}\textbackslash{}end\{aligned\}}
\textbf{inside} that fence to affect alignment of your equations. If you
want each equation to be numbered, then use
\texttt{\textbackslash{}begin\{gather\}\textbackslash{}end\{gather\}} or
\texttt{\textbackslash{}begin\{align\}\textbackslash{}end\{align\}}
fence alone without
\texttt{\textbackslash{}begin\{equation\}\textbackslash{}end\{equation\}}.

If you want no numbering, use \texttt{*} after the outer environment
name, so like
\texttt{\textbackslash{}begin\{equation*\}\textbackslash{}end\{equation*\}}
and
\texttt{\textbackslash{}begin\{align*\}\textbackslash{}end\{align*\}}.
If you want to drop one equation number in the align or gather
environments, use \texttt{\textbackslash{}nonumber}. See examples.

\textbf{cross-refs} If you need to cross-reference the equation number
in the text, you use \texttt{\textbackslash{}label\{equation-name\}} to
give the equation a name. Then cross-reference in text with
\texttt{\textbackslash{}eqref\{equation-name\}}. It is good practice
with amsmath to use \texttt{\textbackslash{}eqref\{\}} instead of
\texttt{\textbackslash{}ref\{\}}. If you don't need to cross-reference
the equation in the text, then you don't need
\texttt{\textbackslash{}label\{\}}. It is only giving the equation a
name; it is not saying ``number this equation''. The numbering or not
comes from having \texttt{*} or not in the outer
\texttt{\textbackslash{}begin\{\}\textbackslash{}end\{\}} or having
\texttt{\textbackslash{}nonumber} on an equation.

\section{Basic split}\label{basic-split}

This is similar behavior to aligned.

\begin{verbatim}
\begin{equation} \label{eq1}
\begin{split}
A & = \frac{\pi r^2}{2} \\
 & = \frac{1}{2} \pi r^2
\end{split}
\end{equation}
\end{verbatim}

\begin{equation} \label{eq1}
\begin{split}
A & = \frac{\pi r^2}{2} \\
 & = \frac{1}{2} \pi r^2
\end{split}
\end{equation}

The cross-ref with \texttt{\textbackslash{}eqref\{eq1\}} to get
\eqref{eq1}.

\section{Basic gathered}\label{basic-gathered}

Gathered gets you one number for all equations with centered equations.

\begin{verbatim}
\begin{equation}  
\begin{gathered}
3(a-x) = 3.5x + a - 1 \\
3a - 3x = 3.5x + a - 1 \\
a = \frac{13}{4}x - \frac{1}{2}
\end{gathered} \label{eq4}
\end{equation}
\end{verbatim}

\begin{equation}  
\begin{gathered}
3(a-x) = 3.5x + a - 1 \\
3a - 3x = 3.5x + a - 1 \\
a = \frac{13}{4}x - \frac{1}{2}
\end{gathered} \label{eq4}
\end{equation}

The cross-ref with \texttt{\textbackslash{}eqref\{eq4\}} to get
\eqref{eq4}.

\section{Basic aligned}\label{basic-aligned}

Same idea as gathered but we want to spec the equation alignment with
\texttt{\&}. \#\# Basic gathered

Gathered gets you one number for all equations with centered equations.

\begin{verbatim}
\begin{equation}  
\begin{aligned}
3(a-x) &= 3.5x + a - 1 \\
3a - 3x + 3a - 3x &= 3.5x + a - 1 
\end{aligned} \label{eq5}
\end{equation}
\end{verbatim}

\begin{equation}  
\begin{aligned}
3(a-x) &= 3.5x + a - 1 \\
3a - 3x + 3a - 3x &= 3.5x + a - 1 
\end{aligned} \label{eq5}
\end{equation}

The cross-ref with \texttt{\textbackslash{}eqref\{eq5\}} to get
\eqref{eq5}.

\section{Basic align}\label{basic-align}

Align (not aligned) gets you each equation numbered with aligning at the
\texttt{\&}.

\begin{verbatim}
\begin{align}  
3(a-x) &= 3.5x + a - 1 \label{eq4a} \\
3a - 3x &= 3.5x + a - 1 \label{eq4b} \\
a &= \frac{13}{4}x - \frac{1}{2} \label{eq4c}
\end{align}
\end{verbatim}

\begin{align}  
3(a-x) &= 3.5x + a - 1 \label{eq4a} \\
3a - 3x &= 3.5x + a - 1 \label{eq4b} \\
a &= \frac{13}{4}x - \frac{1}{2} \label{eq4c}
\end{align}

The cross-ref with \texttt{\textbackslash{}eqref\{eq4a\}} to get
\eqref{eq4a}.

You can put aligned environment within align to get fancier (look at the
raw qmd file for code). \begin{align}
  & {\begin{aligned}
   & 1 + 1 = 2\\
   & 1 * 2 * 3 * 4 * 5 * 6 * 7 = 7!\\
   & 7 + 5 = 12+1-1+1-1+1-1+1-1
  \end{aligned}} \label{A} \\
     12 &= 11 + 1 \label{B}
\end{align} Equations \eqref{A}, \eqref{B}.

Use \texttt{\textbackslash{}nonumber} to drop the number from one line.
Add \texttt{\textbackslash{}label\{\}} if you need to cross-ref the
equations numbers.

\begin{verbatim}
\begin{align}  
3(a-x) &= 3.5x + a - 1  \nonumber \\
3a - 3x &= 3.5x + a - 1 \nonumber \\
a &= \frac{13}{4}x - \frac{1}{2} 
\end{align}
\end{verbatim}

\begin{align}  
3(a-x) &= 3.5x + a - 1  \\
3a - 3x &= 3.5x + a - 1 \nonumber \\
a &= \frac{13}{4}x - \frac{1}{2} 
\end{align}

\section{Basic gather}\label{basic-gather}

gather (not gathered) gets you each equation numbered like align but
equations are all centered.

\begin{verbatim}
\begin{gather}  
3(a-x) = 3.5x + a - 1 \label{eq6a} \\
3a - 3x = 3.5x + a - 1 \label{eq46b} \\
a = \frac{13}{4}x - \frac{1}{2} \label{eq6c}
\end{gather}
\end{verbatim}

\begin{gather}  
3(a-x) = 3.5x + a - 1 \label{eq6a} \\
3a - 3x = 3.5x + a - 1 \label{eq46b} \\
a = \frac{13}{4}x - \frac{1}{2} \label{eq6c}
\end{gather}

The cross-ref with \texttt{\textbackslash{}eqref\{eq6a\}} to get
\eqref{eq6a}.

You can put gathered environment within gather to get fancier.
\begin{gather}
  {\begin{gathered}
   1 + 1 = 2\\
   1 * 2 * 3 * 4 * 5 * 6 * 7 = 7!\\
   7 + 5 = 12+1-1+1-1+1-1+1-1
  \end{gathered}} \label{AA} \\
     12 = 11 + 1 \label{BB}
\end{gather} Equations \eqref{AA}, \eqref{BB}.

\bookmarksetup{startatroot}

\chapter{Fancier equations}\label{fancier-equations}

{(\nextSection)}

MathJax knows most of
\href{https://en.wikibooks.org/wiki/LaTeX/Advanced_Mathematics}{the
fancy amsmath stuff}.

\section{left and right brackets}\label{left-and-right-brackets}

\begin{verbatim}
\begin{equation}
 \left.\begin{aligned}
        B'&=-\partial \times E,\\
        E'&=\partial \times B - 4\pi j,
       \end{aligned}
 \right\}
 \qquad \text{Maxwell's equations}
\end{equation}
\end{verbatim}

\begin{equation}
 \left.\begin{aligned}
        B'&=-\partial \times E,\\
        E'&=\partial \times B - 4\pi j,
       \end{aligned}
 \right\}
 \qquad \text{Maxwell's equations}
\end{equation}

\section{alignat}\label{alignat}

\begin{verbatim}
\begin{alignat}{2}
 \sigma_1 &= x + y  &\quad \sigma_2 &= \frac{x}{y} \\   
 \sigma_1' &= \frac{\partial x + y}{\partial x} & \sigma_2' 
    &= \frac{\partial \frac{x}{y}}{\partial x}
\end{alignat}
\end{verbatim}

\begin{alignat}{2}
 \sigma_1 &= x + y  &\quad \sigma_2 &= \frac{x}{y} \\   
 \sigma_1' &= \frac{\partial x + y}{\partial x} & \sigma_2' 
    &= \frac{\partial \frac{x}{y}}{\partial x}
\end{alignat}

\section{split within gather}\label{split-within-gather}

\begin{verbatim}
\begin{gather*}
a_0=\frac{1}{\pi}\int\limits_{-\pi}^{\pi}f(x)\,\mathrm{d}x\\[6pt]
\begin{split}
a_n=\frac{1}{\pi}\int\limits_{-\pi}^{\pi}f(x)\cos nx\,\mathrm{d}x=\\
=\frac{1}{\pi}\int\limits_{-\pi}^{\pi}x^2\cos nx\,\mathrm{d}x
\end{split}\\[6pt]
\begin{split}
b_n=\frac{1}{\pi}\int\limits_{-\pi}^{\pi}f(x)\sin nx\,\mathrm{d}x=\\
=\frac{1}{\pi}\int\limits_{-\pi}^{\pi}x^2\sin nx\,\mathrm{d}x
\end{split}\\[6pt]
\end{gather*}
\end{verbatim}

\begin{gather*}
a_0=\frac{1}{\pi}\int\limits_{-\pi}^{\pi}f(x)\,\mathrm{d}x\\[6pt]
\begin{split}
a_n=\frac{1}{\pi}\int\limits_{-\pi}^{\pi}f(x)\cos nx\,\mathrm{d}x=\\
=\frac{1}{\pi}\int\limits_{-\pi}^{\pi}x^2\cos nx\,\mathrm{d}x
\end{split}\\[6pt]
\begin{split}
b_n=\frac{1}{\pi}\int\limits_{-\pi}^{\pi}f(x)\sin nx\,\mathrm{d}x=\\
=\frac{1}{\pi}\int\limits_{-\pi}^{\pi}x^2\sin nx\,\mathrm{d}x
\end{split}\\[6pt]
\end{gather*}

\section{Boxes around equations}\label{boxes-around-equations}

\begin{align}
\Aboxed{ f(x) & = \int h(x)\, dx} \\
              & = g(x)
\end{align}




\end{document}
